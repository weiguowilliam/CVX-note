In this section, we prove the linear convergence of Sinkhorn Algorithm in discrete setting case.

\begin{equation}
\left\{
\begin{array}{lr}
D_1^k = \rho_{1.}/(e^{-c/\epsilon}D^{k-1}_2)\\
D_2^k = \rho_{2.}/([e^{-c/\epsilon}]^{T}D_1^k)
\end{array}
\right.
\end{equation}

In this part, we use the \textbf{Hilbert metric}. Let $R^N_{++}$ be the cone of the positive coordinate vector. Then we have the Hilbert metric on this cone as:

\begin{equation}
    \mu(x,y)  \overset{def}{=}  max\limits_{i,j}log(\frac{x_i y_j}{x_j y_i}) \ .
\end{equation}

We make some explanations about this formula here: $\mu$ is non-negative since we can take $i=j$ and get $\mu(x,y)\geq log(1) = 0$. Furthermore, $\mu(x,\lambda x)=0$, so the Hilbert metric must be on $R^n_{++}/R_{>0}$. 
We can also find that $\forall i,j $ we have $\frac{x_i}{y_i} = \frac{x_j}{y_j}$ is a result from the case $\mu(x,y)=0$.
This also implies that the quantity is independent of the index.
Based on the inequality, we can find that the Hilbert metric is a metric on $[S_n]_{++} \overset{def}{=} S_n \cap R_{++}^{n}$. 

Since the Hilbert metric Space is a Metric Space, we have the result that the set $[S_n]++$ endowed with the Hilbert metric is complete.
% https://proofwiki.org/wiki/Hilbert_Sequence_Space_is_Complete_Metric_Space

Since $[S_N]_{++}$ is an open set of $R^n$, we can easily find that the abovementioned theorem is non-trivial. Based on this theorem, we can get the celebrated Birkhoff theorem:

\begin{equation}
    \mu(Ax, Ay) \leq \kappa(A) \mu(x,y) \: \: \forall \: x,y \in R_{++}^n 
\end{equation}

where $A\in R_{++}^{m\times n}$, the constant $\kappa(A)=tanh(\frac{\delta(A)}{4})<1$ and 

\begin{equation}
    \delta(A) = max\limits_{i,j} \mu(Ae_i, Ae_j) = max\limits_{i,j,k,l} log(\frac{A_{ik}A_{jl}}{A_{il}A_{jk}})
\end{equation}

% We show the proof of the Birkhoff theorem here. 

Since we already have the Birkhoff theorem, we will get the linear convergence of Sinkhorn algorithm because the Gibbs kernel matrix is $k=e^{-C_{ijk}/\epsilon}$ whose entries are positive: to show this, we will use the following properties of the Hilbert metric: 
the pointwise multiplication on $R_{++}^n$ ($(x\cdot y)_i = x_i y_i$) as well as inversion $inversion((x^{-1})_i = 1/x_i)$ are isometries for the Hilbert metric. 


Based on the above-mentioned two properties, we next show the linear convergence for the discrete case of Sinkhorn algorithm.

We have two sequences $D_1^k$ and $D_2^k$ generated by the Sinkhorn algorithm. Based on the definition of Hilbert metric, we have:

\begin{equation}
    \mu(D_2^k,D_2^{k+1}) = \mu(1_{m.}/(A^T D^K_1),1_{m.}/(A^T D^{k+1}_1)) = \mu(A^T D^k_1, A^T D_1^{k+1}) \leq \kappa(A^T)\mu(D_1^k,D_1^{k+1})
\end{equation}

We can easily find that if we iterate this argument and use the fact that $\kappa(A^T) = \kappa(A)$, we will have the following formula:

\begin{equation}
    \mu(D_2^k,D_2^{k+1}) \leq \kappa(A)^2 \mu(D_2^k,D_2^{k-1})
\end{equation}



% https://pdfs.semanticscholar.org/687f/4bbff8fd80bb7ce8b78ebbba47fb392b8088.pdf?_ga=2.34674209.1677005450.1606698324-1380262680.1606698324
% We will use the Projective contraction theorem to show the convergence. We prove the Projective contraction theorem here. 

Let $N(P,C)<1$ for some $r$, and let $C$ be complete relative to $\theta(f,g,C)$. Then, for any $f\in C$, the sequence of $fP$ converge 


Denote $\kappa$ as the contraction constants, 
then the rest of the proof follows from standard arguments on contractions. We can also say that the proof here is a direct application of the previous property. 
The discrete Sinkhorn algorithm linearly converges to its unique solution. 

